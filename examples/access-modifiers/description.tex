%%%%%%%%%%%%%%%%%%%% author.tex %%%%%%%%%%%%%%%%%%%%%%%%%%%%%%%%%%%
%
% sample root file for your "contribution" to a contributed volume
%
% Use this file as a template for your own input.
%
%%%%%%%%%%%%%%%% Springer %%%%%%%%%%%%%%%%%%%%%%%%%%%%%%%%%%


% RECOMMENDED %%%%%%%%%%%%%%%%%%%%%%%%%%%%%%%%%%%%%%%%%%%%%%%%%%%
\documentclass[graybox]{svmult}

% choose options for [] as required from the list
% in the Reference Guide

\usepackage{mathptmx}       % selects Times Roman as basic font
\usepackage{helvet}         % selects Helvetica as sans-serif font
\usepackage{courier}        % selects Courier as typewriter font
\usepackage{type1cm}        % activate if the above 3 fonts are
                            % not available on your system
%
\usepackage{makeidx}         % allows index generation
\usepackage{graphicx}        % standard LaTeX graphics tool
                             % when including figure files
\usepackage{multicol}        % used for the two-column index
\usepackage[bottom]{footmisc}% places footnotes at page bottom
\usepackage[utf8]{inputenc}
\usepackage[russian]{babel}
\usepackage[OT1]{fontenc}
\usepackage{amsmath}
\usepackage{amssymb}
\usepackage{caption}

% see the list of further useful packages
% in the Reference Guide

\makeindex             % used for the subject index
                       % please use the style svind.ist with
                       % your makeindex program

%%%%%%%%%%%%%%%%%%%%%%%%%%%%%%%%%%%%%%%%%%%%%%%%%%%%%%%%%%%%%%%%%%%%%%%%%%%%%%%%%%%%%%%%%

\begin{document}


\title*{Контекст как основа модификаторов доступа}
\author{shamcode}
\maketitle

\abstract{Использование понятия "Контекста" упрощает описание модели модификаторов доступа, в частоности private, protected, а также friendly methods}

\section{Модель модификаторов доступа}
\label{sec:1}
Введем несколько определений
\begin{definition}
$F(C)$ - множество методов класса $C$
\end{definition}

\begin{definition}
$a: F(C) \to \{public, protected, private\}$ - модификатор методов
\end{definition}

\begin{definition}
$i: A \to B$ - класс $A$ наследуется от $B$
\end{definition}

\begin{definition}
$[m_1, \dots, m_n]$ - стек вызова, состоящий из последовательного вызова методов $m_1$, $\dots$, $m_n$. 
\end{definition}

Очевидно, что если первый метод в стеке вызова принадлежит классу $C$, то этот публичный метод, т.е.:
\begin{theorem}
\label{first_method_must_be_public}
Если $[m_1, \dots, m_n]$ и $m_1 \in F(C)$, то $a(m_1) = public$
\end{theorem}

Опишем с помощью этих определений модификаторы доступа $private$ и $protected$:

\begin{definition}
\label{private_def}
Для стека вызова $[\dots, m_i, m_{i+1}, \dots]$ верно
\begin{equation*}
\left\{
\begin{aligned}
&m_{i+1} \in F(C) \\
&a(m_{i+1}) = private
\end{aligned}
\right. 
\implies m_i \in F(C)
\end{equation*}
\end{definition}

Иными словами, если текущий метод в стеке вызова приватный и принадлежит классу $C$, то и предыдущий метод в стеке принадлежит классу $C$.


\end{document}
