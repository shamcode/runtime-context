%%%%%%%%%%%%%%%%%%%% author.tex %%%%%%%%%%%%%%%%%%%%%%%%%%%%%%%%%%%
%
% sample root file for your "contribution" to a contributed volume
%
% Use this file as a template for your own input.
%
%%%%%%%%%%%%%%%% Springer %%%%%%%%%%%%%%%%%%%%%%%%%%%%%%%%%%


% RECOMMENDED %%%%%%%%%%%%%%%%%%%%%%%%%%%%%%%%%%%%%%%%%%%%%%%%%%%
\documentclass[graybox]{svmult}

% choose options for [] as required from the list
% in the Reference Guide

\usepackage{mathptmx}       % selects Times Roman as basic font
\usepackage{helvet}         % selects Helvetica as sans-serif font
\usepackage{courier}        % selects Courier as typewriter font
\usepackage{type1cm}        % activate if the above 3 fonts are
                            % not available on your system
%
\usepackage{makeidx}         % allows index generation
\usepackage{graphicx}        % standard LaTeX graphics tool
                             % when including figure files
\usepackage{multicol}        % used for the two-column index
\usepackage[bottom]{footmisc}% places footnotes at page bottom
\usepackage[utf8]{inputenc}
\usepackage[russian]{babel}
\usepackage[OT1]{fontenc}
\usepackage{amsmath}
\usepackage{amssymb}
\usepackage{caption}

\def\definitionname{Определение}%
\def\theoremname{Теорема}

% see the list of further useful packages
% in the Reference Guide

\makeindex             % used for the subject index
                       % please use the style svind.ist with
                       % your makeindex program

%%%%%%%%%%%%%%%%%%%%%%%%%%%%%%%%%%%%%%%%%%%%%%%%%%%%%%%%%%%%%%%%%%%%%%%%%%%%%%%%%%%%%%%%%

\begin{document}


\title*{Контекст как основа модификаторов доступа}
\author{shamcode}
\maketitle

\abstract{Использование понятия "Контекста" упрощает описание модели модификаторов доступа, в частоности private, protected.}
\section{Модель модификаторов доступа}
\label{sec:1}
Введем несколько определений
\begin{definition}
$F(C)$ - множество методов класса $C$
\end{definition}

\begin{definition}
$a: F(C) \to \{public, protected, private\}$ - модификатор методов
\end{definition}

\begin{definition}
$i: A \to B$ - класс $A$ наследуется от $B$
\end{definition}

\begin{definition}
$[m_1, \dots, m_n]$ - стек вызовов, состоящий из последовательного вызова методов $m_1$, $\dots$, $m_n$ таких, что $m_i$ был вызван из $m_{i-1}$.
\end{definition}


\begin{definition}
\label{correct_stack}
Стек вызова  $[m_1, \dots, m_n]$ называется корректным если для него выполняются условия:

\begin{equation}
\label{first_method_public}
m_1 \in F(C) \implies a(m_1) = public
\end{equation}

\begin{equation}
\label{private}
\left\{
\begin{aligned}
&m_{j+1} \in F(C) \\
&a(m_{j+1}) = private
\end{aligned}
\right. 
\implies m_j \in F(C)
\end{equation}

\begin{equation}
\label{protected}
\left\{
	\begin{aligned}
		&m_{j+1} \in F(i(C)) \\
		&a(m_{j+1}) = protected
	\end{aligned}
\right.
\implies
\left[
	\begin{aligned}
		&m_j \in F(C) \\
		&m_j \in F(i(C))
	\end{aligned}
\right.
\end{equation}

\end{definition}

\begin{definition}
Стек вызовов, для которого хотя бы одно из условий \ref{first_method_public}, \ref{private}, \ref{protected} не выполненно будем называть некорректным или стеком вызова с ошибками доступа.
\end{definition}

\section{Свойства корректного стека вызовов}
\label{sec:2}

\begin{theorem}
\label{not_private_before_private}
Для корректного стека $[\dots, m_j, \dots]$ верно:
\begin{equation}
	\left\{	
		\begin{aligned}
			&m_j \in F(C)\\
			&a(m_j) = private
		\end{aligned}
	\right.
	\implies
	\exists m_k: 
	\left\{
		\begin{aligned}
			&k < j \\
			&m_k \in F(C) \\
			&a(m_k) \neq private
		\end{aligned}
	\right.
\end{equation}
\end{theorem}
\begin{proof}
По свойству \ref{private} определения корректного стека \ref{correct_stack}, получаем, что $m_{j-1} \in F(C)$. Если $a(m_{j-1}) \neq private$, то мы получаем искомый $m_k$, т.к $j - 1 < j$.

Если же $a(m_{j-1}) = private$, то применяя свойства \ref{private}, получаем, что $m_{j-2}$ либо является искомым, либо $a(m_{j-2}) = private$.

Последовательно применяя свойства \ref{private} мы либо найдем искомый метод, либо дойдем до $m_1 \in F(C)$. Но по свойству \ref{first_method_public} получаем $a(m_1) = public$. 
Что и требовалось доказать. $\blacksquare$

\end{proof}

Аналогично доказывается:
\begin{theorem}
\label{method_before_protected}
Для корректного стека $[\dots, m_j, \dots]$ верно:
\begin{equation}
	\left\{	
		\begin{aligned}
			&m_j \in F(C)\\
			&a(m_j) = protected
		\end{aligned}
	\right.
	\implies
	\exists m_k: 
	\left\{
    	\begin{aligned}
			&k < j \\
			&a(m_k) = public \\
			&\left[
        		\begin{aligned}
					& m_k \in F(C) \\
   					&\left\{
   						\begin{aligned}
	   						m_j &\in F(C_n) \\
	   						C_n &= i(C_{n-1}) \\
	   						&\vdots \\
	   						C_1 &= i(C)
   						\end{aligned}
   					\right.
				\end{aligned}
			\right.
		\end{aligned}
	\right.
\end{equation}
\end{theorem}

На основе этих двух теорем мы можем доказать общую:
\begin{theorem}
\label{method_before_protected_and_private}
Для корректного стека $[\dots, m_j, \dots]$ верно:
\begin{equation}
	\left\{	
		\begin{aligned}
			&m_j \in F(C)\\
			&a(m_j) \in \{private, protected\}
		\end{aligned}
	\right.
	\implies
	\exists m_k: 
	\left\{
    	\begin{aligned}
			&k < j \\
			&a(m_k) = public \\
			&\left[
        		\begin{aligned}
					& m_k \in F(C) \\
   					&\left\{
   						\begin{aligned}
	   						m_j &\in F(C_n) \\
	   						C_n &= i(C_{n-1}) \\
	   						&\vdots \\
	   						C_1 &= i(C)
   						\end{aligned}
   					\right.
				\end{aligned}
			\right.
		\end{aligned}
	\right.
\end{equation}
\end{theorem}

\section{Оределение контекста вызова}
\label{sec:2}
\begin{definition}
\label{context}
Множество $\theta$ называется контекством вызова для стека вызовов $[\dots, m_j, \dots]$, если 
\begin{equation}
\forall m_j \in F(C_k): C_k \in \theta
\end{equation}
\end{definition}

Перепишем определение \ref{correct_stack} с помощью контекста вызова:
\begin{definition}
\label{correct_stack_with_context}
Стек вызовов $[m_1, \dots, m_n]$ с конекстом выполнения $\theta$ называется корректным если для него выполняются условия:

\begin{equation}
\label{first_method_public_with_context}
m_1 \in F(C) \implies C \in \theta
\end{equation}

\begin{equation}
\label{private_with_context}
\left\{
\begin{aligned}
&m_{j+1} \in F(C) \\
&a(m_{j+1}) = private
\end{aligned}
\right. 
\implies C \in \theta
\end{equation}

\begin{equation}
\label{protected_with_context}
\left\{
	\begin{aligned}
		&m_{j+1} \in F(i(C)) \\
		&a(m_{j+1}) = protected
	\end{aligned}
\right.
\implies
\left[
	\begin{aligned}
		&C \in \theta \\
		&i(C) \in \theta
	\end{aligned}
\right.
\end{equation}

\end{definition}

Докажем следующую теорему:
\begin{theorem}
Если для стека вызов $[m_1, \dots, m_n]$ с конекстом выполнения $\theta$ верно
\begin{equation}
\forall m_j \in F(C_k): a(m_j) \neq private \implies C_k \in \theta
\end{equation}
то он является корректным стеком вызовов.
\end{theorem}
\begin{proof}

\end{proof}
\end{document}
